\section{Discusión}
%Se incluirá aquí un análisis de los resultados obtenidos en la sección anterior (se analizará
%su validez, coherencia, etc.). Deben analizarse como míınimo los ítems pedidos en el
%enunciado. No es aceptable decir que “los resultados fueron los esperados”, sin hacer
%clara referencia a la teoría la cual se ajustan. Además, se deben mencionar los resul-
%tados interesantes y los casos “patológicos” encontrados.





\subsection{Hipótesis 1:}
Con respecto a la hipótesis 1, que establecía que corriendo cualquiera de los 2 métodos implementados(kNN y kNN+PCA) fijando k y $\alpha$, a mayor valor de K mayor porcentaje de tasas, procederemos primero a observar los resultados de kNN+PCA.\\ Comparando las tasas de efectivad resultantes de usar K=2 contra K=10 y K=20, se nota claramente que las de K=2 son menores que las de los otros dos valores de K. Esto es consecuencia principal del uso de la técnica de K-Fold Cross Validation, ya que con un valor K chico nuestra partición de la base de entrenamiento se divide en partes mas grandes, en consecuencia la longitud de la base de test tiende a ser igual a la base de entrenamiento. Dicho en otros términos, tenemos menos elementos en la base de entrenamiento para entrenar nuestro algoritmo y menos permutaciones sobre cuales elementos los consideramos parte del test. Esto es la principal causa de que con K=2 obtengamos los menores valores de la tasa de efectivida. Para que la tasa de efectividad tienda a 1, es necesario contar con una base lo más extensa posible. Y además al aumentar K, tenemos mas combinacion de particiones posibles y un entrenamiento más profundo. En este caso, la disminucion de la misma no se ve afectada por el valor que tome k y alpha. \\Hasta aqui la hipótesis se confirmaría, pero al analizar y comparar las tasas de K=10 contra K=20, observamos que aca si incide el valor que tome k. Para k con valores chicos, se ve una diferencia de la tasa cuando saltamos de K=10 a K=20, pero a medida que k aumenta, las tasas empiezan a comportarse de manera similar, teniendo vaivenes para algunos k, en el que la tasa de K=10 le gana a K=20 o viceversa. Ante esta situación, decidimos observar más detalladamente los datos de la experimentación y logramos ver que para k grandes y K=20, la muestra de las tasas de cada partición posee una mayor varianza, es decir que los valores tienen mayores fluctuaciones en comparación a la media de la muestra. Realizando un análisis más intensivo, deducimos que esto se debe a que a mayor valor de K, tengo mas elementos de entrenamiento, y a su vez a mayor valor de k, tengo más elemento que entrarán en la definición de vecinos más cercanos, pudiendo aqui encontrar elementos que no son del mismo label que el elemento en estudio y así alternar notablemente la tasa de efectividad en cada una de las particiones.\\
Como último dato, el valor de $\alpha$ no incide en las tasas de efectividad entre diferentes K, es decir, para un $\alpha$ fijo, a medida que se aumenta el valor de K, aumenta la tasa.\\
Con respecto al método de kNN, la elección de los K, se comporta de la misma manera y tiene las mismas consecuencias que en el método de kNN+PCA.
Finalmente se pordría concluir que la hipótesis queda parcialmente verificada, ya que la única parte que se refuto de la misma fue la parte de mantener k constante.

\textbf{Análisis de tiempo de computo al variar K:}
 En el método de kNN+PCA, claramente se incrementa el tiempo de cómputo a medida que aumenta el valor de K, esto se debe a que el cálculo de la matriz de covarianza del entrenamiento domina gran parte del tiempo del metodo. Dicha matriz se calcula de la siguiente forma: por ejemplo, si K=2, la matriz de entranamiento va a tener un tamaño de 21000x784, su transpuesta de 784x21000, entonces la matriz de covarianza se calcula multiplicando la transpuesta por la matriz de entrenamiento resultando siempre una matriz de 784x784. Para este caso de K, haremos 2 multiplicaciones de matrices de 784x21000 y 21000x784. Ahora bien, si suponemos que K=10, el procedimiento es el mismo, salvo que ahora, mi base de entrenamiento va a ser mas grande, por ende, su matriz también que este caso va a tener tamaño de (42.000-42.000/10)x784 = 37800x784. Para este caso de K, haremos 10 multiplicaciones de matrices de 784x37.800 y 37.800x784. Al aumentar el valor de K, además de hacer más multiplicaciones por tener más particiones, las matrices aumentan de tamaño por lo explicado anteriormente. Para K=20 es el mismo procedimiento.\\
En el método de kNN, por ejemplo para K=2, voy a recorrer 2 veces la base de entrenamiento de 21000 dígitos para un base de test con la misma cantidad de dígitos. Para k=10, voy a recorrer 10 veces la base de train de 37.800 dígitos para una base de test de 4.200 digitos. Realizando un análisis más exhaustivo, para el primer caso de K=2, estariamos recorriendo 21.000 veces la base de 21.000 digitos, todo eso 2 veces lo que resultaria en 2*21.000*21.000 = 822.000.000 iteraciones. Para el caso de K=10, estariamos recorriendo 4200 veces la base de 37800 digitos, todo eso 10 veces lo que resultaria en 10*4.200*37.800 = 1.587.600.000 iteraciones, lo que implacarŕia un 93$\%$ más de iteraciones. Para K=20 pasa lo mismo.\\



\subsection{Competencia Kaggle}
Para realizar nuestro submission en la página de Kaggle, a la competencia de Digit Recognizer, modificamos levemente el codigo de nuestro programa. El mismo se encuentra en la carpeta Kaggle. Las modificaciones se deben a que en este caso, la base de test proporcionada por la página no tiene labels, no utilizamos el método de K-Fold Cross Validation, la funcion kNN ahora tiene que devolver las predicciones de los dígitos, entre otras cosas.\\
Luego de toda la experimentación concluímos que los mejores parámetros, en cuanto tiempo y eficacia, para correr el programa son algún k y $\alpha$ cercanos a:
\begin{itemize}
\item $\alpha$ = 50
\item $k$ = 2
\end{itemize}

A continuación probaremos con valores cercanos a estos parámetros para estudiar la efectividad que nos proporciona el sistema de Kaggle.
\begin{itemize}
	\item Con k=2 y $\alpha$=50, tuvimos una efectividad de 0.96686
	\item Con k=3 y $\alpha$=50, tuvimos una efectividad de 0.96857
	\item Con k=3 usando solo kNN, tuvimos una efectividad de 0.97200
	\item Con k=4 y $\alpha$=50, tuvimos una efectividad de 0.97286
\end{itemize}