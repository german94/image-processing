\section{Conclusiones}
%Esta sección debe contener las conclusiones generales del trabajo. Se deben mencionar
%las relaciones de la discusión sobre las que se tiene certeza, junto con comentarios
%y observaciones generales aplicables a todo el proceso. Mencionar también posibles
%extensiones a los métodos, experimentos que hayan quedado pendientes, etc.
Muchos de los problemas de la vida real que quisieramos resolver por computadora son de naturaleza continua, esto significa que las herramientas matemáticas que se utilizan para tratarlos, trabajan con números reales. La computadora, al trabajar con aritmética finita, necesita que planteemos el problema a resolver de una manera discreta. En este trabajo, utilizar el método de aproximación por diferencias finitas, nos permitió modelar un problema continuo de manera discreta. De esta manera, pudimos adaptar la ecuación del calor a nuestra discretización del parabrisas, fijando distintos valores para el parámetro $h$.\newline
Al realizar variaciones en el $h$ pudimos observar como impacta en la calidad de los resultados. Pero, por si solo esta variación no actuaba como un factor determinante. Como se pudo observar en la experimentación 1. En algunos casos, aunque se variaba la granularidad de un sistema, la temperatura en el punto crítico seguia siendo la mismo. Concluyendo que un factor importante recaía además en las dimensiones del parabrisas. Ya que como se mencionó cuando el punto crítico dependia de más de un punto. Se presentaban variaciones entre la temperatura del punto crítico para distintas discretizaciones. Si bien, no podemos concluir que sí el punto crítico depende de solo uno entonces, no variará su temperatura (dada las limitaciones en la cantidad de experimentos desarollados). Si podemos pensar que existe una mayor probabilidad de que esto suceda, que si trabajamos con dimensiones que requieren que nuestro punto crítico dependa de la temperatura de un promedio de puntos. A su vez, cuando se producían cambios, un factor influyente fue el radío de las sanguijuelas. Estas producían un cambio mas o menos bruscos, en la distribución de las temperaturas, a medida que se variaba su radío. Atribuyendole a granularidades mayor o igual que los radios promedios de las sanguijuelas, resultados peores en comparación a los casos contraríos.  
Además experimentamos obtener una relación adecuada entre el tiempo de cómputo y la calidad de la solución obtenida. Esta búsqueda no suele ser fácil, sobre todo cuando queremos evaluar la calidad de una solución, dado que desconocemos la solución real al problema, puede llegar a suceder que las soluciones varién mucho según el tipo de granularidad elegida. Una de las cosas que pudimos averiguar mediante la experimentación, es que para las instancias del problema cuyos radios de las sanguijuelas sean muy pequeños, es probable que las soluciones varíen mucho dependiendo de la granularidad elegida. Esto no sucede si los radios de las sanguijuelas son grandes y, mucho menos si hay alguna sanguijuela que actua de manera directa en el punto crítico. También pudimos ver que el tiempo de cómputo de los algoritmos propuestos para la resolución del problema, no está dominado por la cantidad de sanguijuelas de una instancia del mismo. Experimentos que quedaron pendientes que hubiera servido para aumentar la validez de estas afirmaciones consistirían en realizar comparaciones en los tiempos de cómputo de resolver instancias sin sanguijuelas contra instancias con sanguijuelas, variando nuevamente el radio y granularidad. Por ejemplo, una instancia sin sanguijuelas debería tardar más en resolverse que una instancia con sanguijuelas de radios grandes (en el contexto del tamaño de la discretización). Sin embargo, no deberían existir grandes diferencias entre el tiempo de cómputo consumido para resolver una instancia sin sanguijuelas y otra instancia con sanguijuelas de radios muy pequeños.\\
En materia a lo referido en las experimentaciones de eliminarSanguijuela, aprendimos lo importante que es abordar un tema con una base teórica fuerte, como por ejemplo con la utilización de la formula Sherman-Morrison. Dicha formula nos permitió aumentar la eficiencia del algoritmo eliminarSanguijuela más de lo que nos imaginábamos. Como consecuencia de esto, nos trajo curiosidad pensar ideas y variantes a este ecuación que sean aplicables al caso general. En un futuro nos gustaría poder desarrollarlas y experimentar para ver si es posible llevar la resolución del sistema al mínimo posible de complejidad temporal. La curiosidad sobre la modificación de dicha escuación, también vino por parte de la experimentación que realizamos cambiando la ganularidad, verificamos que al disminuir la misma, la distancia de los puntos de la discretizacion disminuye, haciendo bajar la probabilidad de encontrar sanguijuelas unitarias, en estos casos el algoritmo con la formula propuesta en el enunciado no era mucho más eficiente que el algoritmo original.
