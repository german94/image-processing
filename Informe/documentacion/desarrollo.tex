\section{Desarrollo}
%Deben explicarse los métodos numéricos que utilizaron y su aplicación al problema
%concreto involucrado en el trabajo práctico. Se deben mencionar los pasos que si-
%guieron para implementar los algoritmos, las dificultades que fueron encontrando y la
%descripción de cómo las fueron resolviendo. Explicar también cómo fueron planteadas
%y realizadas las mediciones experimentales. Los ensayos fallidos, hipótesis y conjeturas
%equivocadas, experimentos y métodos malogrados deben figurar en esta sección, con
%una breve explicación de los motivos de estas fallas (en caso de ser conocidas).
La idea del siguiente experimento, es tratar de ver como se relaciona el tiempo de cómputo necesario para resolver el sistema de ecuaciones planteado, con la calidad de la solución encontrada. Es importante para esto, decir a qué nos referimos cuando hablamos de calidad. Es necesario para responder esta pregunta, recordar que nuestro objetivo es calcular la temperatura en cierto punto del parabrisas. Para lograr esto, lo que hacemos es discretizar la superficie (porque estamos trabajando con aritmética finita) y pasar de una superficie que tiene infinitos puntos a una representación de dicha superficie con una cantidad finita de puntos. Ahora bien, cuando realizamos la discretización y pasamos de un problema en el espacio continuo a uno en el espacio discreto estamos perdiendo información. Entonces cuando hablamos de calidad, nos referimos justamente a cuanta información estamos perdiendo, como costo de modelar en problema de manera discreta.
\par El experimento se planteó para trabajar con algunas instancias del problema y ver como varía la temperatura en el punto crítico obtenida y el tiempo de cómputo, utilizando el algoritmo de Eliminación Gaussiana (modo 0), en función de la granularidad de la discretización (h). 
\newline \par Ahora que sabemos qué es lo que se intenta determinar en el experimento, vamos a tratar de pensar qué es lo que debería pasar (y luego contrastarlo con los resultados para ver si efectivamente es lo que sucede) a grandes rasgos. 
\begin{enumerate}
\item En cuanto al tiempo de cómputo, sabemos que como vamos a estar utilizando el algoritmo de Eliminación Gaussiana, la cota teórica va a ser $O(n^3)$. El sistema de ecuaciones con el que trabajaremos va a ser de $n'$x$n'$ donde $n' = (b/h - 1)$x$(a/h - 1)$ siendo $a$ (ancho del parabrisas), $b$ (alto del parabrisas), $h$ (granularidad) parámetros del problema. Podemos ver como el tamaño del sistema (y por lo tanto el tiempo de cómputo ya que es una función del tamaño) depende directamente de $h$, de manera tal que si $h$ es chico, entonces el tamaño es grande. Esto nos dice entonces que, dado una instancia del parabrisas (con sus dimensiones y sanguijuelas), cuanto más chico sea $h$, más tiempo se va a tardar en resolver el problema.
\item Debido a que perdemos información al discretizar el problema porque estamos trabajando con un espacio discreto, cuando en realidad el problema es de naturaleza continua, parece razonable pensar que cuanto más chico sea $h$, la solución obtenida va a estar más cerca de la real.
\item Nuestra última hipótesis se desprende de alguna manera de la que enunciamos en el punto anterior. Si al aumentar el tamaño de nuestra discretización estamos acercándonos más a la verdadera solución del problema y las temperaturas de los puntos dependen de las sanguijuelas, parece razonable pensar que haciendo esto tenemos menos chances de descartar sanguijuelas, ya que nuestra discretización es "más adecuada".
\end{enumerate}


