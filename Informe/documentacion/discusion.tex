\section{Discusión}
%Se incluirá aquí un análisis de los resultados obtenidos en la sección anterior (se analizará
%su validez, coherencia, etc.). Deben analizarse como míınimo los ítems pedidos en el
%enunciado. No es aceptable decir que “los resultados fueron los esperados”, sin hacer
%clara referencia a la teoría la cual se ajustan. Además, se deben mencionar los resul-
%tados interesantes y los casos “patológicos” encontrados.
\par En los experimentos realizados, vimos como efectivamente el tiempo de cómputo aumenta a medida que el $h$ se achica, ya que el tamaño de la discretización es más grande y, como cada punto de la discretización representa una incógnita en nuestro sistema de ecuaciones, vamos a tener que resolver un sistema más grande. Por lo tanto, nuestra hipótesis 1. planteada en el Desarrollo de que el valor de $h$ impacta de manera directa en el tiempo de cómputo es correcta y bien reflejada en nuestros experimentos.
\par Lo que no se ve reflejado en los experimentos, es lo que dicen las hipótesis 1 y 2. Como ambas están relacionadas, vamos a tratarlas de forma conjunto. En la sección de Resultados, podemos ver que el primer experimento no entra en conflicto con lo que dicen estas hipótesis, pero sí el segundo y el tercero. Esto no es casualidad. La diferencia entre el primer experimento y los otros dos radica esencialmente en los radios de las sanguijuelas ya que, omo explicamos en los resultados, los radios se reducen considerablemente en los experimentos 1 y 2. Si consideramos una sanguijuela $S_i$ cuyo radio $r_i$, cuanto más chico sea $r_i$ más chances existen de que $S_i$ sea una sanguijuela unitaria o, en el peor caso, que sea descartada. 