\section{Conclusiones}
%Esta sección debe contener las conclusiones generales del trabajo. Se deben mencionar
%las relaciones de la discusión sobre las que se tiene certeza, junto con comentarios
%y observaciones generales aplicables a todo el proceso. Mencionar también posibles
%extensiones a los métodos, experimentos que hayan quedado pendientes, etc.

Podemos concluir que el método de knn utilizando la norma dos como distancia entre dígitos tiene un margen de error para reconocer a los mismos, lo cual hace que en algunas situaciones no se pueda utilizar, ya que no es aceptable una equivocación. El error de este ultimo método se debe en parte a la cercanía en norma dos de dígitos distintos, pero también en parte a los errores numéricos que se propagan al hacer operaciones aritméticas en la computadora. Este hecho se pone en evidencia al obtener que con el método de pca se obtienen tasas de reconocimiento mas altas. 

Este ultimo método resulta muy efectivo para el reconocimiento de dígitos. En teoría deberían obtenerse tasas menores, ya que se descarta la información menos relevante cuando se reduce la dimensión. Al tratar con menos cantidad de coordenadas para las imágenes, no solo toma un tiempo considerablemente menor, sino que también arrastra menos error numérico, y en consecuencia obtenemos tasas mayores para el reconocimiento de los dígitos.

 
