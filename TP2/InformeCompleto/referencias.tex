%\section{Referencias}
%Es importante incluir referencias a libros, artículos y páginas de Internet consultados
%durante el desarrollo del trabajo, haciendo referencia a estos materiales a lo largo del
%informe. Se deben citar también las comunicaciones personales con otros grupos.


\begin{thebibliography}{1}
\bibitem{Autovectores }Faires, J. D. and Burden, R. L. Análisis Numérico, 7rd ed. Teorema 9.10 y Corolario 9.11. Pág 555, año 2011.
\bibitem{Método de la potencia}Faires, J. D. and Burden, R. L. Análisis Numérico, 7rd ed. Pág 560, año 2011.	
\bibitem{Eliminación Gaussiana}Faires, J. D. and Burden, R. L. Análisis Numérico, 7rd ed. Teorema 9.15. Pág. 570, año 2011.
\end{thebibliography} 