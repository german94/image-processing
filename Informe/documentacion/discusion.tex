\section{Discusión}
%Se incluirá aquí un análisis de los resultados obtenidos en la sección anterior (se analizará
%su validez, coherencia, etc.). Deben analizarse como míınimo los ítems pedidos en el
%enunciado. No es aceptable decir que “los resultados fueron los esperados”, sin hacer
%clara referencia a la teoría la cual se ajustan. Además, se deben mencionar los resul-
%tados interesantes y los casos “patológicos” encontrados.
\subsection{Hipótesis planteadas sobre calidad/tiempo de cómputo}
\par En los experimentos realizados, vimos como efectivamente el tiempo de cómputo aumenta a medida que el $h$ se achica, ya que el tamaño de la discretización es más grande y, como cada punto de la discretización representa una incógnita en nuestro sistema de ecuaciones, vamos a tener que resolver un sistema más grande. Por lo tanto, nuestra hipótesis 1. planteada en el Desarrollo de que el valor de $h$ impacta de manera directa en el tiempo de cómputo es correcta y bien reflejada en nuestros experimentos.
\par Lo que no se ve reflejado en los experimentos, es lo que dicen las hipótesis 3 y 4. Como ambas están relacionadas, vamos a tratarlas de forma conjunta. En la sección de Resultados, podemos ver que las mediciones efectuadas en el primer conjunto de instancias (primera tabla) no entran en conflicto con lo que dicen estas hipótesis, pero sí las que se encuentran en la segunda y tercer tabla. Esto no es casualidad. La diferencia entre el primer conjunto de instancias y los otros dos, radica esencialmente en los radios de las sanguijuelas ya que como explicamos en los resultados, los radios se van reduciendo considerablemente. Sea  $S_i$ una sanguijuela cuyo radio es $r_i$, cuanto más chico sea $r_i$ más chances existen de que $S_i$ sea una sanguijuela unitaria o, en el peor caso, que sea descartada. Esto se debe a que si $S_i$ cae justo en un punto de nuestra discretización, a menos que $r_i = 0$, van a haber más chances de que abarque solo ese punto. Pero si $S_i$ \textbf{no cae} en un punto de nuestra discretización, como $r_i$ es chico podría suceder que entonces $r_i$ no alcance a \textbf{ningún} punto de nuestra discretización, descartando así a $S_i$. Y, como es de esperarse, el hecho de que $S_i$ caiga o no en un punto de nuestra discretización, no depende de que tan chico o grande es $h$, si no de que las coordenadas de la posición de $S_i$ sean múltiplos o no de $h$.
\par Con lo dicho anteriormente podemos afirmar entonces que, por un lado es cierto que cuanto más grande es nuestra discretización más nos acercamos a poder modelar el problema con mayor precisión en cuanto a la propagación del calor (una forma de convencerse de esto matemáticamente, es observar que en la ecuación del calor, $h\to 0$). Pero por otro lado, dado que nuestro problema tiene sanguijuelas con un radio variable que puede ser arbitrariamente chico (dentro de los límites de la computadora) y coordenadas reales, para llegar a una solución razonable es necesario tener en cuenta todas las sanguijuelas y esto no necesariamente depende del tamaño de $h$.
\par Faltaría hablar sobre lo que dice la hipótesis 2. Por los experimentos realizados con instancias de 40x40 (cuyos resultados se encuentran en las últimas dos tablas del experimento relación tiempo-calidad de cómputo), viendo la primera tabla del experimento podemos ver claramente que la hipótesis no se cumple. Esto es así porque los tiempos tomados de la instancia con 100 sanguijuelas son significativamente menores a los tiempos tomados de la instancia con una sanguijuela. Pero esto tampoco significa que valga lo contrario, es decir que tampoco podemos afirmar que "si hay una menor cantidad de sanguijuelas entonces el tiempo de cómputo es mayor". De hecho, en la segunda tabla del experimento con instancias de 40x40, podemos observar que esto tampoco se cumple ya que ahora, los tiempos son muy similares para la instancia de 100 sanguijuelas y la de una sanguijuela. A este punto, es fácil ver que lo que está sucediendo es que, el tiempo de cómputo en realidad está variando en función de el radio de las sanguijuelas. Cuanto mayor es el radio de la sanguijuela, menor es el tiempo de cómputo. Es por esto que también, cuando el radio de una sanguijuela es grande y toca al punto crítico, como es poco probable que la sanguijuela sea descartada al variar la granularidad (debido al tamaño del radio), la temperatura en el punto crítico va a ser igual al valor de la temperatura de la sanguijuela y difícilmente va a variar. En cuanto al tiempo de cómputo, básicamente lo que sucede es que al tener sanguijuelas con radio grande, van a haber muchos puntos de nuestra discretización con una temperatura constante (proveniente de alguna sanguijuela actuando sobre ellos) y por lo tanto, los puntos cuya temperatura sea desconocida y tengamos que calcular van a ser unos pocos.