\section{Introducción Teórica}
%Contendrá una breve explicación de la base teórica que fundamenta los métodos involu- crados en el
%trabajo, junto con los métodos mismos. No deben incluirse demostraciones de propiedades ni
%teoremas, ejemplos innecesarios, ni definiciones elementales (como por ejemplo la de matriz
%simétrica). En vez de definiciones básicas es conveniente citar ejemplos de bibliografía adecuada.
%Una cita vale más que mil palabras.
%


Se plantea el problema de que el parabrisas de una nave está siendo atacado por sanguijuelas mutantes. Dicho ataque consiste en aplicar altas temperaturas constantes sobre la superficie del mismo, con el objetivo de lograr romperlo, para poder lograr un ataque más mortífero. La superficie del parabrisas donde se aplica el calor es circular (la sopapa de ataque es circular). \newline
Para defenderse de estos ataques Guybrush cuenta con un sistema de refrigeración que aplica temperaturas de -100ºC a los bordes del parabrisas. El parabrisas se romperá si alcanza una temperatura mayor o igual a los 235ºC en el punto central (llamaremos a este punto, \textit{punto crítico}).\newline
Si el sistema de refrigeración no es suficiente para salvar el parabrisas, se puede utilizar un arma para destruir una sanguijuela que de ser posible salve el parabrisas y reduzca lo mayor posible la temperatura en el punto critico (llamaremos a dicha sanguijuela \textit{la mejor}). 
En principio deberemos representar al parabrisas, el cual está compuesto por infinitos puntos. Sabemos que no es posible representar en una computadora los infinitos puntos del mismo, por lo que se utilizará cierta discretización.
Si bien la temperatura en los puntos en el estado estacionario satisface la siguiente ecuación.\\

\begin{equation}
\frac{\delta ^2 T(x,y)}{\delta x^2}+\frac{\delta ^2 T(x,y)}{\delta y^2}=0 
\end{equation}\\

Al discretizar los puntos, la ecuación anterior para obtener temperaturas en cada punto del parabrisas continuo, se transforma en la siguiente ecuación por diferencias finitas.
\begin{equation}
t_{ij} = \frac{t_{i-1,j} + t_{i+1,j} + t_{i,j-1} + t_{i,j+1}}{4}
\end{equation}\\
Es decir que en el parabrisas discretizado, la temperatura en cada punto se calcula como el promedio de la temperatura de los puntos vecinos (los puntos que estan arriba, abajo, izquierda y derecha).\newline
Esto planteara nuevos problemas que seran análizados por ejemplo como manejar la aritmética finita de las computadoras para obtener las temperaturas mas precisas. O, si es que influye, como lo hace el útilizar distintas granularidades.\newline
Como ademas debemos ser capaces de dar la temperatura de todos los puntos de la discretización. Para hacerlo, plantearemos el parabrisas discretizado como un sistema de ecuaciones lineales. Haciendo uso de la caracteristicas que va a tener el mismo, ser un sistema Bandas. 
 Y para poder resolver el mismo, se utilizaran tanto la técnica de Eliminación Gaussian, como la factorización LU (siempre tratando de aprovechar las características del sistema).\newline
Como otro de los problemas es salvar el parabrisas (si es posible) eliminando \textit{la mejor} sanguijuela. Se probarara eliminando una por una, recalculando todo el sistema y  obteniendo las nuevas temperaturas en el punto crítico. Pero, como computacionalmente puede resultar costoso rearmar todo el sistema. Experimentaremos calcular, las nuevas temperaturas, aplicando la fórmula de Sherman–Morrison cuando sea posible. Evaluando si mediante esta técnica obtenemos un mejor rendimiento temporal que cuando no se la aplica:(*)
\begin{equation}
	(A+ uv^t)^{-1} \ =\ A^{-1} - \frac{ A^{-1} u v^t A^{-1} }{1+v^t A^{-1}u}.\label{eq:sm}
\end{equation} 
\textit{Siendo A una matriz inversible, y u, $v^t$ dos vectores.}   
   
