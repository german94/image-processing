\section{Discusión}
%Se incluirá aquí un análisis de los resultados obtenidos en la sección anterior (se analizará
%su validez, coherencia, etc.). Deben analizarse como míınimo los ítems pedidos en el
%enunciado. No es aceptable decir que “los resultados fueron los esperados”, sin hacer
%clara referencia a la teoría la cual se ajustan. Además, se deben mencionar los resul-
%tados interesantes y los casos “patológicos” encontrados.


\subsection{Experimento 1}
Hipótesis1: Apartir de los experimentos realizados podemos concluir que la Hipótesis1 es falsa. Debido a diversos mótivos. En el caso número 1, variamos notablemente el valor de las granuladidades con las cuales trabajamos, yendo desde el valor 2 hasta el 10. Y en todos los casos obtuvimos la misma temperatura en el punto crítico. Ademas como se puede observar en \textit{figure 1.a, figure 1.b y figure 1.c}, los puntos restantes poseen una distribución de temperaturas muy similar entre cada Gráfico. En este caso, la temperatura en el punto critico se mantuvo constante entre cada representación.
En el caso número 2, se varío la granuladidad y los valores se modificaron (si bien representa un caso particular, donde las sanguijuelas son mayormente unitarias. sigue siendo un caso posible). Podemos observar que el valor del punto critico en la $figure$ $2.d$ es menor que en la de $2.c$, la cual es menor a su vez que en la obtenida en la $figure$ $2.b$  (entre cada gráfico fuimos disminuyendo el valor de la granularidad). Sin embargo, cuando probamos con h = 0.5, la temperatura en el punto crítico para este caso fue de 350.69248\hspace{-1.5mm}$\phantom{a}^{\circ}$c (ver $figure$ $2.a$) siendo menor que la obtenida para una granularidad de valor 1 ($figure$ $2.b$). Falseando nuestra hipótesis. \newline
El origen de pensar en esta hipótesis se basó en la idea de que al discretizar el sistema por valores de h mas grandes obtenemos menos puntos y la distancia entre ellos es mayor. Asi, si entre dos puntos $A_{ij}$, $A_{i+1,j}$ (de un parabrisas A, discretizado), existía una distancia de valor $h'$ (valor de la granularidad). Y una sanguijuela actua sobre la posicion $i'$ $j'$ con radio $r$ (en el parabrisas sin discretizar) con $j$ $=$ $h'*j'$ e $i*h'$ $<$ $i'$ $+$ $r$ $<$ $(i+1)*h'$. Entonces esa sanguijuela se omitiria (ya que no afecta a ningún punto de la discretización) obteniendo un sistema con una sanguijuela menos y por lo tanto un punto de calor menos. Que si se trabajase con un $h''$, $h''$ $<$ $h'$ e $i'*h''$ $>=$ $i'*r$ $\vee$ $i'*h''$ $<=$ $i'$ $+$ $r$. Al observar este comportamiento ahora fue necesario analizar el por qué de este hecho (agregar si ya hicieron este caso sino explicar).\newline \newline

Hipótesis2: Si bien apartir del experimento anterior podemos ver que no se cumple dicha hipótesis. No podemos dar por totalmente erronea a la misma. Sino, que llegamos a la conclusión de que depende el caso con el que se trabaje. Ya que a diferencia del caso 1. En los otros gráficos (PAGINAS LAS QUE QUEDEN) se puede observar que varían las temperaturas para distintas granularidades. Aún más en la figure 2.d se detalla que el punto crítico (y se puede observar) tiene una temperatura de -100\hspace{-1.5mm}$\phantom{a}^{\circ}$c mientras que para el mismo sistema pero con otra discretizacion alcanza una temperatura de  350.69248\hspace{-1.5mm}$\phantom{a}^{\circ}$c (ver $figure$ $2.a$ ). Siendo la mayor diferencia de temperatura que se registró entre los distintos casos (el por qué de este hecho lo desarrollaremos a lo largo de la hipótesis3 e hipótesis4). Dado que él único caso en el que no se produce un cambio en las temperaturas es el uno. Podemos concluir que se debe a las características propias que presenta el primer caso y que los restantes no. Como tratarse de sistema con sanguijuelas de distintos tamaños. También ser de la forma PARxPAR. Ya que de esta forma el punto crítico solo depende un solo punto de la discretización y es el central. Por lo que si hay una sanguijuela, cuyo radio afecta en este punto y es lo sufientemente grande para permanecer con las granularidades que se utilicen entonces siempre vamos a obtener la misma temperatura en el punto crítico. \newline \newline


Hipótesis3: Trabajamos con una parabrisas de forma IMPARxIMPAR en las PAGINAS LAS QUE SEAN, (segundo y tercer caso del experimento número uno) Por hipóstesis, estos deberian ser los casos donde la temperatura del punto crítico varíe mas entre cada discretizacion utilizada. Y para el primer caso lo es, como ya se menciono se en este sistema se presentó la mayor diferencia. Sin embargo, para el mismo sistema con sanguijuelas de radio mayor (vease caso PAG) la temperatura cambió pero, no tan bruscamente, de hecho en porcentaje es muy similar al caso número 4 (pag). Donde las temperaturas en el punto crítico y en el resto, como puede verse, se mantienen muy similares. Entonces hay otro factor que influye además de que la matriz sea IMPARxIMPAR. Por lo que la hipótesis se podría mejorar diciendo que si se trabaja con una matrix IMPARxIMPAR o un lado PAR y el otro IMPAR (y sanguijuelas de radio variado, hecho analizado en la hipótesis 4). Se van a obtener mayores diferencias que utilizando otras dimensiones con sanguijuelas de radio variado.\newline \newline


Hipótesis4: Como ya hicimos referencia anteriormente los cambios de temperatura más notable se presentaron en el caso 2 (PAG). Y fue donde trabajamos con un predominio de sanguijuelas unitarias. Este cambio no solo se nota en el punto crítico sino que como se puede ver entre la $figure$ $2.a$ y la $figure$ $2.c$ varias puntos afectados por sanguijuelas se pierden entre cada experimentación, aquellos con temperaturas no mayor a los 400\hspace{-1.5mm}$\phantom{a}^{\circ}$c, se pierden en la siguiente experimentacion. Solo permanecen aquellos en los que el radio de accion es mayor a 0. Luego en la $figura$ $2.d$ se perdieron todos los puntos sobre los cuales actuaban las sanguijuelas. De esta forma, y pese a que la granularidad no varío tanto entre los distintos experimentos; Como por ejemplo entre la $figure$ $4.a$ y la $figure$ $4.d$ donde la granularidad va de 1 a 9 o  incluso el mismo sistema pero  con sanguijuelas de mayor tamaño caso 3 (PAG). Podemos concluir que la hipotesis (para estos casos) es verdadera. Más aún, podemos concluir en general que variar las granularidades puede llegar a afectar el resultado de la distribucion de las temperaturas, según el tipo de dimensiones del parabrisas. Pero, que este no es el único factor de distorsión, sino que también esta dado por el radio de las sanguijuelas, afectando en mayor medida si el mismo es de tamaño muy bajo.  


\subsection{Experimento 2}
\subsubsection{Hipótesis planteadas sobre calidad/tiempo de cómputo}
\par En los experimentos realizados, vimos como efectivamente el tiempo de cómputo aumenta a medida que el $h$ se achica, ya que el tamaño de la discretización es más grande y, como cada punto de la discretización representa una incógnita en nuestro sistema de ecuaciones, vamos a tener que resolver un sistema más grande. Por lo tanto, nuestra hipótesis 1. planteada en el Desarrollo de que el valor de $h$ impacta de manera directa en el tiempo de cómputo es correcta y bien reflejada en nuestros experimentos.
\par Lo que no se ve reflejado en los experimentos, es lo que dicen las hipótesis 3 y 4. Como ambas están relacionadas, vamos a tratarlas de forma conjunta. En la sección de Resultados, podemos ver que las mediciones efectuadas en el primer conjunto de instancias (primera tabla) no entran en conflicto con lo que dicen estas hipótesis, pero sí las que se encuentran en la segunda y tercer tabla. Esto no es casualidad. La diferencia entre el primer conjunto de instancias y los otros dos, radica esencialmente en los radios de las sanguijuelas ya que como explicamos en los resultados, los radios se van reduciendo considerablemente. Sea  $S_i$ una sanguijuela cuyo radio es $r_i$, cuanto más chico sea $r_i$ más chances existen de que $S_i$ sea una sanguijuela unitaria o, en el peor caso, que sea descartada. Esto se debe a que si $S_i$ cae justo en un punto de nuestra discretización, a menos que $r_i = 0$, van a haber más chances de que abarque solo ese punto. Pero si $S_i$ \textbf{no cae} en un punto de nuestra discretización, como $r_i$ es chico podría suceder que entonces $r_i$ no alcance a \textbf{ningún} punto de nuestra discretización, descartando así a $S_i$. Y, como es de esperarse, el hecho de que $S_i$ caiga o no en un punto de nuestra discretización, no depende de que tan chico o grande es $h$, si no de que las coordenadas de la posición de $S_i$ sean múltiplos o no de $h$.
\par Con lo dicho anteriormente podemos afirmar entonces que, por un lado es cierto que cuanto más grande es nuestra discretización más nos acercamos a poder modelar el problema con mayor precisión en cuanto a la propagación del calor (una forma de convencerse de esto matemáticamente, es observar que en la ecuación del calor, $h\to 0$). Pero por otro lado, dado que nuestro problema tiene sanguijuelas con un radio variable que puede ser arbitrariamente chico (dentro de los límites de la computadora) y coordenadas reales, para llegar a una solución razonable es necesario tener en cuenta todas las sanguijuelas y esto no necesariamente depende del tamaño de $h$.
\par Faltaría hablar sobre lo que dice la hipótesis 2. Por los experimentos realizados con instancias de 40x40 (cuyos resultados se encuentran en las últimas dos tablas del experimento relación tiempo-calidad de cómputo), viendo la primera tabla del experimento podemos ver claramente que la hipótesis no se cumple. Esto es así porque los tiempos tomados de la instancia con 100 sanguijuelas son significativamente menores a los tiempos tomados de la instancia con una sanguijuela. Pero esto tampoco significa que valga lo contrario, es decir que tampoco podemos afirmar que "si hay una menor cantidad de sanguijuelas entonces el tiempo de cómputo es mayor". De hecho, en la segunda tabla del experimento con instancias de 40x40, podemos observar que esto tampoco se cumple ya que ahora, los tiempos son muy similares para la instancia de 100 sanguijuelas y la de una sanguijuela. A este punto, es fácil ver que lo que está sucediendo es que, el tiempo de cómputo en realidad está variando en función de el radio de las sanguijuelas. Cuanto mayor es el radio de la sanguijuela, menor es el tiempo de cómputo. Es por esto que también, cuando el radio de una sanguijuela es grande y toca al punto crítico, como es poco probable que la sanguijuela sea descartada al variar la granularidad (debido al tamaño del radio), la temperatura en el punto crítico va a ser igual al valor de la temperatura de la sanguijuela y difícilmente va a variar. En cuanto al tiempo de cómputo, básicamente lo que sucede es que al tener sanguijuelas con radio grande, van a haber muchos puntos de nuestra discretización con una temperatura constante (proveniente de alguna sanguijuela actuando sobre ellos) y por lo tanto, los puntos cuya temperatura sea desconocida y tengamos que calcular van a ser unos pocos.

\subsection{Experimento 3}

Mediante la experimentación realizada, se logro verificar la hipótesis (1). Realmente, el tiempo de computo disminuye notablemente al usar el algoritmo modificado con Sherman-Morrison en casos donde haya sanguijuelas unitarias. Los resultados mostraron que cuanto más sanguijuelas unitaria haya, menor va a ser el tiempo de computo del mismo, comparado con el algoritmo original.\\
Esto significó un acercamiento práctico a lo expuesto teóricamanete en la sección del desarrollo, es decir, que se pudo constatar la justificación teórica por la cual es mejor aplicar Sherman-Morrison.\\
Un caso para resaltar, es cuando no hay ninguna sanguijuela unitaria, los 2 algoritmos tienen un tiempo de computo similar, hasta en algunos caso el algoritmo modificado con Sherman-Morrison tarda unos segundos más que el original, debido a que tiene más saltos condicionales y evalua más expresiones.\\
\\
En cuanto a la hipótesis (2), aunque asumimos predicados muy fuertes para que la experimentación sea representativa y no librada al azar (como que las sanguijuelas estén ubicadas en puntos múltiplos de la granularidad por ejemplo), se logro verificar lo que intuitivamente pensamos. Al disminuir la granularidad, la distancia entre 2 puntos cualesquiera se achica y la cantidad de puntos de la discretización aumenta. Esto implica que la probabilidad de encontrar sanguijuelas unitarias también disminuye, ya que es más probable, que al haber más puntos, cada sanguijuelas toque más de un punto del parabrisa.


