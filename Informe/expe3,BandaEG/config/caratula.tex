
\begin{figure}[ptb]
\includegraphics[scale=0.30]{logo.jpg}\hspace{6cm}
\includegraphics[scale=0.90]{logo_dc.jpg}
\end{figure}

\newcommand{\autores}{Armagno, Balbachán, More, Pinzón}

%Datos de la caratula
\materia{M\'etodos Num\'ericos}
\titulo{Trabajo Pr\'actico 1}
\subtitulo{No creo que a él le gustará eso}
\hspace{6cm}
\integrante{Armagno, Julián Adrián}{377/12}{julian.armagno@gmail.com}
\integrante{Balbachan, Alexis}{}{}
\integrante{More, Ángel}{}{}
\integrante{Pinzón, Germán}{}{}
\resumen{En este trabajo estudiaremos algoritmos para resolver sistemas de ecuaciones para problemas
reales mediante una discretización. Se expondrán técnicas para obtener la temperatura en un punto crítico de un parabrisa usando eliminación Gaussiana y Factorización LU, aprovechando las matrices Banda, y la formula Sherman-Morrison  Al final
del trabajo, se llegarán a conclusiones sobre lo descubierto.}
\palabrasClave{Eliminación Gaussiana. Factorización LU. Punto Crítico. Matrices Banda. Sistemas de
  ecuaciones. Sherman-Morrison.}
\hypersetup{%
 % Para que el PDF se abra a página completa.
 pdfstartview= {FitH \hypercalcbp{\paperheight-\topmargin-1in-\headheight}},
 pdfauthor={\autores},
 pdfsubject={TP1}
}

\parskip=5pt % 10pt es el tamaño de fuente

% Pongo en 0 la distancia extra entre ítemes.
\let\olditemize\itemize
\def\itemize{\olditemize\itemsep=0pt}

% Acomodo fancyhdr <- Creo que es el encabezado de pagina
\pagestyle{fancy}
\thispagestyle{fancy}
\addtolength{\headheight}{1pt}
\lhead{M\'etodos Num\'ericos - 1$^{er}$ Cuatr. 2015}
\rhead{TP1 - \autores}
\cfoot{\thepage}
\renewcommand{\footrulewidth}{0.4pt}




%Pagina de titulo e indice
\thispagestyle{empty}

\maketitle
\tableofcontents

\newpage

