
\begin{figure}[ptb]
\includegraphics[scale=0.30]{logo.jpg}\hspace{6cm}
\includegraphics[scale=0.90]{logo_dc.jpg}
\end{figure}

\newcommand{\autores}{Armagno, More, Pinzón, Porto}

%Datos de la caratula
\materia{M\'etodos Num\'ericos}
\titulo{Trabajo Pr\'actico 2}
\subtitulo{Reconocimiento de dígitos.}
\hspace{6cm}
\integrante{Armagno, Julián}{377/12}{julian.armagno@gmail.com}
\integrante{More, Ángel}{931/12}{angel\_21\_fer@hotmail.com}
\integrante{Pinzón, Germán}{475/13}{pinzon.german.94@gmail.com}
\integrante{Porto, Jorge}{376/11}{cuanto.p.p@gmail.com}
\resumen{En este trabajo pondremos en práctica distintos algoritmos para reconocer una cierta cantidad de dígitos manuscritos. Se trabajará con una base train, a la cual le realizaremos particiones para entrenar los algoritmos. Se implementará el metodo de kNN(k-Nearest Neighbors). Debido a que este es sensible a la dimensión de los objetos a considerar, además se implementará el método de Análisis de Componentes Principales para reducir el tamaño de dichos objetos. Se llevarán a cabo experimentaciones para poder determinar los parámetros óptimos para cada método. Al final del trabajo, se llegarán a conclusiones sobre lo descubierto.}
\palabrasClave{Learning Machine. kNN. Análisis de Componentes Principales. Auto-Valores. Auto-Vectores. Método de las Potencias. K-fold cross validation}
\hypersetup{%
 % Para que el PDF se abra a página completa.
 pdfstartview= {FitH \hypercalcbp{\paperheight-\topmargin-1in-\headheight}},
 pdfauthor={\autores},
 pdfsubject={TP1}
}

\parskip=5pt % 10pt es el tamaño de fuente

% Pongo en 0 la distancia extra entre ítemes.
\let\olditemize\itemize
\def\itemize{\olditemize\itemsep=0pt}

% Acomodo fancyhdr <- Creo que es el encabezado de pagina
\pagestyle{fancy}
\thispagestyle{fancy}
\addtolength{\headheight}{1pt}
\lhead{M\'etodos Num\'ericos - 1$^{er}$ Cuatr. 2015}
\rhead{TP1 - \autores}
\cfoot{\thepage}
\renewcommand{\footrulewidth}{0.4pt}




%Pagina de titulo e indice
\thispagestyle{empty}

\maketitle
\tableofcontents

\newpage

